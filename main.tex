\documentclass[11pt,English]{article}
\usepackage[utf8]{inputenc}
\usepackage{amsmath}
\usepackage[bottom]{footmisc}


% Slides 
% 
% 1) Authors names and affiliations
% 2) Title abstract keywords
% 3) Importance

\title{
Green's Theorem\\
    \large Historical Origins and Analytical Applications\\
    \small MATH 147 Final Project\\
    \small University of Kansas, Dept. of Mathematics
}
\author{
    Atkins, Thomas\\
    \texttt{thomas.atkins@ku.edu}
    \and
    Mills, Garrett\\
    \texttt{glmdev@ku.edu}
    \and
    Weng, QiTao\\
    \texttt{wengqt@ku.edu}
}
\date{December 2019}

\begin{document}

\maketitle
\begin{abstract}
    In which the authors investigate the historical origins and several mathematical applications of the commonly known Green's theorem. Discovered by George Green in the late 1820s, this theorem provides a relationship between the line integral of a particular curve and the surface integral of its enclosed region. Green's theorem is closely related to the divergence theorem, and is simply a specific case of the more general Stoke's theorem. Beyond basic applications to flux and surface integrals, Green's theorem can be reverse applied to calculate difficult-to-evaluate area calculations. It also plays an integral role (pun intended) in the proof of other important theorems such as Cauchy's.
\end{abstract}

\section{Introduction}

Green's theorem is commonly defined as follows.\footnote{"Section 5-7: Green's Theorem" - Paul Dawkins, Lamar University - 02-22-2019. (http://tutorial.math.lamar.edu/Classes/CalcIII/GreensTheorem.aspx)} Let $C$ be a simple, smooth, closed, positive curve and $D$ the region enclosed by said curve. Assume $P'$, $Q'$ are continuous. Then, the following relationship holds:
$$
\int_C{ P dx + Q dy } = \iint_D{ \left( \frac{\partial Q}{\partial x} - \frac{\partial P}{\partial y} \right) dA }
$$

\end{document}
